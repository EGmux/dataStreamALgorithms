\section{Notes related to probability concepts}

\subsection{Sketch}

The techinique is to apply a "linear projection on the fly" that converts high dimensional data into smaller dimensional data and then extract a quantity of interest
\cite{10-jhu1} in the lower dimension.
\\
Can also call the smaller dimensional data as summary.

\subsubsection{Formal definition}

\begin{math}
	V:=(x_0,x_1,...x_n)
	\\
	h:=\text{any hash function}
	\\
	x:=\text{any vector}
	\\\\
	V^n → V^m
	\\
	x →  h(x_i)
	\\
\end{math}

\begin{lstlisting}[language=Haskell]
  Sketch :: [a] -> (a -> a) -> [a]
  Sketch x h = foldl h x
  \end{lstlisting}
\textit{obs: h deals per type of x not the whole vector}
\\
\textit{obs2: we use foldl and not map because a reduction in the size of the vector happens and map deal with same size vectors.}
\cite{hemmanga}

\subsection{Estimation}

Before we can discuss estimation we must explain what is the process of statical inference. well inference is guessing something based on previous information
\\
note the word \textbf{guessing} there's no guarantee the right answer will be reached but an "accurate approximation" nonetheless.

Thus estimation is a technique to achieve statical inference, the idea is that estimation allow us to guess the value of parameters that model some probability distribution

What to know about the quality of estimations:

\subsubsection{$\mu$ as symbol for confidence interval}

the idea here is that the estimation can fall in a range, generally between 95\%-99\% that guarantees a confidence level

\subsubsection{n as symbol for sample size}

Well for each estimation we assume a minimum ammount of points in the data also called sampling that will allow the estimation process to happen and fall in the confidence interval

\cite{notesEstimation}

\subsection{Estimator}

A fancy name for someting very simple an function that receives a set of samples and output a estimate that is a parameter to model the distribution
here some fancy names for categorizing these functions

\begin{itemize}
	\item Biased: underestimation or overestimation
	\item Efficient: small variance
	\item Invariant: hard to change even after transformations, imagine apply a increase in each sampled datapoint the estimation would stay the same
	\item Shrinkage: raw estimation
	\item Sufficient: enough samples to extract a confidence level compliant estimation
	\item  unbiased: not over/under estimation
\end{itemize}
\textit{OBS: statistic is not the same as estimation, estimation is a way to acquire an statistic, there are other ways as well}
\cite{howto}

\subsection{$(\epsilon, \delta)$-approximation}

Another notational way to say that estimation error is related to ϵ and the confidence interval of that information is related to the δ parameter
\\
Thus one could say with ϵ < 0.2 we guarantee with δ > 0.9 that such estimate is accurate.
\\\\
\textbf{map absolute value of error to ϵ}
\\
\textbf{map confidence interval to δ}
