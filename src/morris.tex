\section{The Morris Counter}

The Morris counter is an algorithm created to allow approximate calculations of values in devices that can't represent
such numbers due to physical limitations, amount of bits available.

\subsection{What is approximate counting?}

The concept of approximate counting is deducing the total number of events that occurred but storing as minimal events as possible
to allow such deduction.
\\
To compute the error parameters we mostly look at the standard deviation, that give us a measure of accuracy/trustworthiness.

Effectively we can double, triple , n-increase the tracked value but storing a fraction of it size, indeed a very powerful technique.

\subsection{So why the Morris counter?}

Might not be obvious but a considerable difference in the error according to the sample size implies less guarantees, thus a method
that guarantees for the average case a stable margin of error, that is won't oscillate is a good value proposition.

