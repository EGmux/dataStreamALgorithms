Notes based on the following resources:
\\
\cite{blogMorris}

\section{The Morris Counter}

The Morris counter is an algorithm created to allow approximate calculations of values in devices that can't represent
such numbers due to physical limitations, amount of bits available.

\subsection{What is approximate counting?}

The concept of approximate counting is deducing the total number of events that occurred but storing as minimal events as possible
to allow such deduction.
\\
To compute the error parameters we mostly look at the standard deviation, that give us a measure of accuracy/trustworthiness.

Effectively we can double, triple , n-increase the tracked value but storing a fraction of it size, indeed a very powerful technique.

\subsection{So why the Morris counter?}

Might not be obvious but a considerable difference in the error according to the sample size implies less guarantees, thus a method
that guarantees for the average case a stable margin of error, that is won't oscillate is a good value proposition.

we use the following equation as an starting point to understand the algo

\begin{equation}
	v_n = ln(1+n)
\end{equation}
\\
the choice for logarithm is due to the fact that v increases much faster than n, thus low amount of events  tracked guarantee better approximations
and average case error is constant.

However the value of the amount tracked,v, is not an integer and must be rounded up/down.

\subsection{How to determine the rounding}

\subsubsection{Cost innacuracy}

Assume an increase in \textbf{r}, either the value of v will be undershoot or overshoot, we need an heuristic to deal with this, thus the concept of a marker, \textbf{d}
to help with this we also use a rng approach to generate the parameter \textbf{r} and if r is larger or smaller than the mark d, then do the following:
\begin{equation}
	d = \frac{1}{n_{v+1} - n_v}
	\label{eq:parameter d}
\end{equation}
\\\\
if larger: increase the counter
\\
if smaller: keep current value
\\\\
proof of this is here: \cite{morris}
\\\\
then we get to this final equation
\begin{equation}
	v = \frac{log(n+1)}{log(2)}
	\label{eq:final computation of tracked value}
\end{equation}

\subsection{The main advantage of this approach}

Saving space, the amount of bits required to be tracked is $log(n)$ and bits implies binary counting that is with n bits we can track $2^n$ states that eventually map
to numbers, thus a binary morris counter actually is $O(log(log(n)))$ in spatial complexity, because is not dealing with integers, but bits.
